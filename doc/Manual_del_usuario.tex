\documentclass[a4paper,14.49998pt]{article}
\usepackage[latin1]{inputenc}
\usepackage{t1enc}
\usepackage[pdftex]{graphicx}
\usepackage[spanish]{babel}
\usepackage{multicol}
\usepackage{fancybox}
\usepackage{fancyhdr}
\usepackage[pdftex,usenames,dvipsnames]{color}
\usepackage{listings}
\usepackage{ucs}
\usepackage{textcomp}
\lstset{
	tabsize=4,
	rulecolor=,
	language=matlab,
        basicstyle=\scriptsize,
        upquote=true,
        aboveskip={1.5\baselineskip},
        columns=fixed,
        showstringspaces=false,
        extendedchars=true,
        breaklines=true,
        prebreak = \raisebox{0ex}[0ex][0ex]{\ensuremath{\hookleftarrow}},
        frame=single,
        showtabs=false,
        showspaces=false,
        showstringspaces=false,
        identifierstyle=\ttfamily,
        keywordstyle=\color[rgb]{0,0,1},
        commentstyle=\color[rgb]{0.133,0.545,0.133},
        stringstyle=\color[rgb]{0.627,0.126,0.941}
}
\pagenumbering{arabic}
\pagestyle{fancy}
\fancyhead[L]{Arqui}
\fancyhead[C]{curso 2011 primer cuatrimestre}
\fancyhead[R]{TP Especial}
\title{\Huge{Manual del Usuario\\\vspace{15mm}Arnix\\\vspace{15mm}Modo protegido con GRUB}}
\begin{document}
\maketitle
\vspace{70mm}
\large{\underline{Autores:}} 
\begin{center}
\begin{tabular}{l r}
\emph{\author{Axel Wassington}} & Legajo: \emph{50124}\\
\emph{\author{Horacio Miguel Gomez}} & Legajo: \emph{50825}\\
\emph{\author{Tom�s Mehdi}} & Legajo: \emph{51014}
\end{tabular}
\end{center}
\pagebreak
\tableofcontents
\pagebreak

\section{Manual de uso}
Esta seccion esta destinada a proporcionar informaci�n sobre las funcionalidades
del sistema y explicar su modo de uso.
\subsection{Booteo}
El sistema es booteable, por lo que para su inicio solamente se requiere que el CD
se encuentre insertado y la lectora posea la mayor prioridad de booteo. Luego se
mostrar� la pantalla de GRUB. Se puede presionar
enter para proceder, o de lo contrario GRUB continuar� la carga del sistema tras un
n�mero prestablecido de segundos, visible en el ultimo rengl�n de la pantalla. Una vez realizado 
el booteo, se cargar� el shell. A partir de
aqu� se puede comenzar a utilizar los comandos provistos por el shell.
\subsection{Comandos de shell}
La siguiente lista indica la forma de usar a los distintos comandos de el shell.
Es importante destacar que todos las llamadas son case sensitive. En caso
de ingresar un comando que no esta contemplado por el shell se imprime un
mensaje de error y se devuelve el control al usuario.
\subsubsection{rename}
Cambia el nombre de usuario en la consola.
\subsubsection{help}
Muestra todo los comandos disponibles de la shell.
\subsubsection{isodd}
Informa si el n�mero ingresado es par o impar.
\subsubsection{test}
Acepta los parametros printf y scanf, y imprime algunos tests de los diferentes tipos de data
input type.
\subsubsection{getCPUspeed}
Imprime en pantalla la velocidad a la que trabaja el CPU.
\subsubsection{clear}
Limpia la pantalla.
\subsubsection{echo}
Recive una cadena de caracteres luego del nombre del comando e imprime dicha cadena.
\subsubsection{exit}
Cierra el shell y pide que ingreses un nuevo nombre de usuario para la shell.
\pagebreak
\end{document}