\documentclass[a4paper,14.49998pt]{article}
\usepackage[latin1]{inputenc}
\usepackage{t1enc}
\usepackage[pdftex]{graphicx}
\usepackage[spanish]{babel}
\usepackage{multicol}
\usepackage{fancybox}
\usepackage{fancyhdr}
\usepackage[pdftex,usenames,dvipsnames]{color}
\usepackage{listings}
\usepackage{ucs}
\usepackage{textcomp}
\lstset{
	tabsize=4,
	rulecolor=,
	language=matlab,
        basicstyle=\scriptsize,
        upquote=true,
        aboveskip={1.5\baselineskip},
        columns=fixed,
        showstringspaces=false,
        extendedchars=true,
        breaklines=true,
        prebreak = \raisebox{0ex}[0ex][0ex]{\ensuremath{\hookleftarrow}},
        frame=single,
        showtabs=false,
        showspaces=false,
        showstringspaces=false,
        identifierstyle=\ttfamily,
        keywordstyle=\color[rgb]{0,0,1},
        commentstyle=\color[rgb]{0.133,0.545,0.133},
        stringstyle=\color[rgb]{0.627,0.126,0.941}
}
\pagenumbering{arabic}
\pagestyle{fancy}
\fancyhead[L]{Arqui}
\fancyhead[C]{curso 2011 primer cuatrimestre}
\fancyhead[R]{TP Especial}
\title{\Huge{Informe\\\vspace{15mm}Arnix\\\vspace{15mm}Modo protegido con GRUB}}
\begin{document}
\maketitle
\vspace{70mm}
\large{\underline{Autores:}} 
\begin{center}
\begin{tabular}{l r}
\emph{\author{Axel Wassington}} & Legajo: \emph{50124}\\
\emph{\author{Horacio Miguel Gomez}} & Legajo: \emph{50825}\\
\emph{\author{Tom�s Mehdi}} & Legajo: \emph{51014}
\end{tabular}
\end{center}
\pagebreak
\tableofcontents
\pagebreak


\section{Decisiones e implementaci�n del sistema}
\subsection{C�digo}
Se utilizo un mix de assembler con C por decisi�n de la catedra. Es muy importante aclarar
este punto, por que varias instrucciones solo pueden hacerse desde assembler y para simplificar
la codificacion de usa C llamando a assembler o assembler llamando a C.
\subsection{Compilaci�n, linkedici�n y ejecuci�n}
Reemplazamos el compila por un makefile y el arma por un build. El build puede llamarse con el parametro
clean o no. Si se llama con el parametro se hace un make clean y make. Sino solo se hace el make. Luego
de estos en los dos casos si el c�digo compila se abre el bochs.
\subsection{Pantalla}
\subsection{Tabla global de descriptores}
\subsection{Interrupciones}

\section{Funcionamiento del comando getCPUspeed}
Muestra la frecuencia de trabajo del CPU, usando la funcion RDTSC(Read Time-Stamp Counter) de 
assembler la cual retorna la cantidad de instrucciones realizada hasta el momento desde el
inicio del procesador y el PIT(Programable Interval Timer). El PIT es un periferico conectado
a la IRQ0 del master PIC. Utilizando estos elementos podemos obtener una cantidad de
instrucciones en un intervalo de tiempo. La forma de hacerlo es pidiendo un RDTSC, dejar pasar
un tiempo fijo y volver a pedir un RDTSC. El tiempo fijo lo generamos con una cantidad coherente
de timer ticks, para ello no debe ser muy peque�a. Ya teniendo estos datos solo falta hacer 
simples cuentas que nos devolveran la velocidad del CPU en MHz.

\section{INT80h}
Similar a la INT80h de Unix/Linux; la INT80h de Arnix segun el valor en el registro 
EAX elige una instruccion.Las instrucciones que puede realizar son las siguientes:
\begin{enumerate}
\item Con el valor 3 en EAX hace un read usando los valores de EBX, ECX y EDX. En estos regristros debe
estar el tama�o de lo que se va a leer, el source buffer y un file descriptor(de donde leer) respectivamente.

\item Con el valor 4 en EAX hace un write usando los valores de EBX, ECX y EDX. En estos regristros debe
estar el tama�o de lo que se va a escribir, el source buffer y un file descriptor(donde esribir) 
respectivamente.

\item Con el valor 5 en EAX hace un rdtsc guardando el valor en el registro EBX.
\end{enumerate}
\pagebreak

\section{Referencias}
Esta secci�n detalla las distintas fuentes de informaci�n utilizadas para
el desarrollo del TP Especial. Es importante destacar que son las mismas fuentes
enviadas en un mail previo a la entrega.
\subsection{Interrupciones}
El manejo de interrupciones es similar al usado en el siguiente tutorial:
\begin{center}
http://www.jamesmolloy.co.uk/tutorial\_html/4.-The\%20GDT\%20and\%20IDT.html\\
\end{center}
Nos pareci� interesante la opci�n de crear un wrapper para las idt y luego desde
C simplemente asignar handlers a las convenientes, un wrapper se encarga de que a 
C le lleguen los parametros.Tambi�n creimos importante tener las entrys para las 
primeras 31 exceptions del procesador, para evitar resets si por ejemplo dividimos por cero.

\subsection{Pantalla}
Nos basamos en la implementaci�n de Linux, la pantalla puede recibir scape chars para 
limpiar la pantalla, o imprimir en colores. Esto era conveniente debido a que al 
utilizar la int80, no necesitamos parametros extra, para estas funcionalidades extra.

\subsection{Reboot}
Luego de probar soluciones sucias, como hacer que el procesador triple-faultee y 
se reinicie la pc. Encontr�mos la solucion de enviar la se�al de reset desde el 
controlador de teclado en:
\begin{center}
http://wiki.osdev.org/Reboot
\end{center}
\subsection{Paginas web utilizadas}
En general leimos mucho de:
\begin{center}http://wiki.osdev.org/Main\_Page ( y sus foros )\\
http://www.jamesmolloy.co.uk/tutorial\_html/index.html\\
http://www.osdever.net/tutorials/view/brans-kernel-development-tutorial\\
\end{center}
\end{document}
