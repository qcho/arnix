\documentclass[a4paper,14.49998pt]{article}
\usepackage[latin1]{inputenc}
\usepackage{t1enc}
\usepackage[pdftex]{graphicx}
\usepackage[spanish]{babel}
\usepackage{multicol}
\usepackage{fancybox}
\usepackage{fancyhdr}
\usepackage[pdftex,usenames,dvipsnames]{color}
\usepackage{listings}
\usepackage{ucs}
\usepackage{textcomp}
\lstset{
	tabsize=4,
	rulecolor=,
	language=matlab,
        basicstyle=\scriptsize,
        upquote=true,
        aboveskip={1.5\baselineskip},
        columns=fixed,
        showstringspaces=false,
        extendedchars=true,
        breaklines=true,
        prebreak = \raisebox{0ex}[0ex][0ex]{\ensuremath{\hookleftarrow}},
        frame=single,
        showtabs=false,
        showspaces=false,
        showstringspaces=false,
        identifierstyle=\ttfamily,
        keywordstyle=\color[rgb]{0,0,1},
        commentstyle=\color[rgb]{0.133,0.545,0.133},
        stringstyle=\color[rgb]{0.627,0.126,0.941}
}
\pagenumbering{arabic}
\pagestyle{fancy}
\fancyhead[L]{Arqui}
\fancyhead[C]{curso 2011 primer cuatrimestre}
\fancyhead[R]{TP Especial}
\title{\Huge{Informe\\\vspace{15mm}Arnix\\\vspace{15mm}Modo protegido con GRUB}}
\begin{document}
\maketitle
\vspace{70mm}
\large{\underline{Autores:}} 
\begin{center}
\begin{tabular}{l r}
\emph{\author{Axel Wassington}} & Legajo: \emph{50124}\\
\emph{\author{Horacio Miguel Gomez}} & Legajo: \emph{50825}\\
\emph{\author{Tom�s Mehdi}} & Legajo: \emph{51014}
\end{tabular}
\end{center}
\pagebreak
\tableofcontents
\pagebreak


\section{Decisiones e implementaci�n del sistema}
\subsection{C�digo}
Se utilizo un mix de assembler con C por decisi�n de la catedra. Es muy importante aclarar
este punto, por que varias instrucciones solo pueden hacerse desde assembler y para simplificar
la codificacion del TP se  usa C llamando a assembler o assembler llamando a C.
\subsection{Compilaci�n, linkedici�n y ejecuci�n}
Reemplazamos el compila por un makefile y el arma por un build. El build puede llamarse con el parametro
clean o no. Si se llama con el parametro se hace un make clean y make. Sino solo se hace el make. Luego
de estos en los dos casos si el c�digo compila se abre el bochs.
\subsection{Pantalla}
Se utilizaron ANSI escape squences, estos son chars incrustados en el texto que se utiliza para
controlar el formato, color y otras opciones de salida en las terminales de video en formato 
texto. Nosotros implementamos los siguientes ANSI scape characters:
\begin{center}
Esc[2J           Borra la pantalla y mueve el cursor a (line 0, column 0)
Esc[\#;\#;...m 	Cambia el modo de graficos segun los siguientes atributos:
\end{center}
\begin{enumerate}
  \item   Text attributes
   \subitem 0	All attributes off
 \subitem 1	Bold on
 \subitem 4	Underscore (on monochrome display adapter only)
 \subitem 5	Blink on
  
 \item Foreground colors    
 \subitem 30 Black          
 \subitem 31 Red             
 \subitem 32 Green          
 \subitem 33 Yellow         
 \subitem 34 Blue           
 \subitem 35 Magenta        
 \subitem 36 Cyan           
 \subitem 37 White          
\item Background colors
\subitem  40 Black
\subitem 41 Red
\subitem  42 Green
\subitem 43 Yellow
\subitem  44 Blue
\subitem  45 Magenta
\subitem  46 Cyan
\subitem  47 White 

 
\end{enumerate}
\subsection{Interrupciones}
Para manejar las interrupciones creamos dos estructuras, una de ellas describe una interrupt gate
y la otra tiene un puntero a un array de interrupts handlers. Tambien creamos una estructura que
esta integrada por todos los registros del micro, la que nos permite obtener los datos necesarios
para saber como ejecutar las distintas interrupciones. Las INT80h es una interrupcion importante,
mediante la cual se hace un paso desde el userspace al kernel space para utilizar algunos codig�s.
Mas adelante en este informe se detalla el uso del a INT80h. Tambi�n es importante mencionar
que remapeamos todas las interrupciones para que no se pisen con las exepciones(posisiones 0-31 
de la idt) lo que nos facilito mucho la programacion.

\subsection{Shell}
Al iniciar la shell se agregan todas los comandos, esto se hace utilizando una estructura 
que tiene un puntero a funci�n y el nombre del comando. El maximo tama�o de un comando son 
1000 caracteres, si te pasas ignora los las siguitenes entradas. Cada 500 caracteres se 
hace un flush del stream y no te permite borrar los 500 caracteres anteriores. Se puede 
ingresar un argumento entre comillas para que lo considere como un solo argumento. A diferencia 
de las shells que normalmente utilizamos (Unix/Linux), nuestro shell ignora los espacios 
del principio y del final y los espacios multiples entre medio de distintos argumentos. 


\subsection{Division del Kernel y user space}
Hicimos una muy buena implementacio para que la division de estas dos partes sea notoria. Estos
se logr� dividiendo en diferentes carpetas los distintos archivos relacionados con cada espacio.
Todas las llamas a sistema se hicieron a travez de funciones codificadas en systemcalls.asm.

\subsection{Funcionamiento del comando getCPUspeed}
Muestra la frecuencia de trabajo del CPU, usando la funcion RDTSC(Read Time-Stamp Counter) de 
assembler la cual retorna la cantidad de instrucciones realizada hasta el momento desde el
inicio del procesador y el PIT(Programable Interval Timer). El PIT es un periferico conectado
a la IRQ0 del master PIC. Utilizando estos elementos podemos obtener una cantidad de
instrucciones en un intervalo de tiempo. La forma de hacerlo es pidiendo un RDTSC, dejar pasar
un tiempo fijo y volver a pedir un RDTSC. El tiempo fijo lo generamos con una cantidad coherente
de timer ticks, para ello no debe ser muy peque�a. Ya teniendo estos datos solo falta hacer 
algunas cuentas que nos devolveran la velocidad del CPU en MHz. Dichas cuentas son para hacer
el calculo mas presiso, hacer la resta entre los 2 valores de RDTSC en orden de 
obtencion(el primero menos el segundo) multiplicado por 18(cantidad de ticks para 
alcanzar 1 segundo) dividido la cantidad de ticks que utilizamos y a eso sumado la resta en orden
de obtencion de los RDTSC dividido la cantidad de ticks multiplicado por 5 que representa el .2 de
 los 18.2 ticks que hay en un segundo. Esta cuenta nos devuelve la cantidad de instrucciones por segundo. 
Este valor dividido $1024*1024$ nos da la velocidad en MHz de CPU.

\subsection{INT80h}
Similar a la INT80h de Unix/Linux; la INT80h de Arnix segun el valor en el registro 
EAX elige una instruccion.Las instrucciones que puede realizar son las siguientes:
\begin{enumerate}
\item Con el valor 3 en EAX hace un $read$ usando los valores de EBX, ECX y EDX. En estos regristros debe
estar el tama�o de lo que se va a leer, el source buffer y un file descriptor(de donde leer) respectivamente.

\item Con el valor 4 en EAX hace un $write$ usando los valores de EBX, ECX y EDX. En estos regristros debe
estar el tama�o de lo que se va a escribir, el source buffer y un file descriptor(donde esribir) 
respectivamente.

\item Con el valor 5 en EAX hace un $cpuspeed$ guardando la cantidad de
$IPS$(instrucciones por segundo) en donde apunta el registro EBX.
\end{enumerate}


\subsection{Printf}
Se codeo este comando lo mas parecido al de la libreria estandar de C, pero a pesar de eso
no todas las opciones estan en ella. Las opciones del printf que se hicieron son las m�s
utilizadas; imprimir un $int, string, decimal a octal, decimal a hexa, un caracter, 
unsigned int$. El printf utiliza putchar.c la cual utiliza la funcion \_\_write.

\subsection{Teclado}
Se crearon dos mapeos de teclado, uno es para cuando el shift esta apretado y el otro cuando no.
Hay cuatro vectores de listeners que se utilizan para los distintos estados del teclado
seg�n el keypress de las teclas $CTRL$ y $ALT$. Cuando se aprieta el $BLOCK MAYUS$ se activa una
variable, mediante la cual el teclado actua simplemente sumando un ascii para que los valores
cambien de lowerCase a upperCase o de upperCase a lowerCase. Al apretar a la vez $CTRL+ALT+DEL$
el sistema se reinicia, mandando un valor determinado a cierta direcci�n del teclado.

\subsection{IN-OUT}
Tenemos un vector de diez lugares en los cuales estan los distintos tipos de $in-out$. En la
primera posici�n del vector ($index = 0$) esta el standard $in$ y en la segunda posici�m el standard
$out$
\pagebreak
\section{Referencias}
Esta secci�n detalla las distintas fuentes de informaci�n utilizadas para
el desarrollo del TP Especial. Es importante destacar que son las mismas fuentes
enviadas en un mail previo a la entrega.
\subsection{Interrupciones}
El manejo de interrupciones es similar al usado en el siguiente tutorial:
\begin{center}
http://www.jamesmolloy.co.uk/tutorial\_html/4.-The\%20GDT\%20and\%20IDT.html\\
\end{center}
Nos pareci� interesante la opci�n de crear un wrapper para las idt y luego desde
C simplemente asignar handlers a las convenientes, un wrapper se encarga de que a 
C le lleguen los parametros.Tambi�n creimos importante tener las entrys para las 
primeras 31 exceptions del procesador, para evitar resets si por ejemplo dividimos por cero.

\subsection{Pantalla}
Nos basamos en la implementaci�n de Linux, la pantalla puede recibir scape chars para 
limpiar la pantalla, o imprimir en colores. Esto era conveniente debido a que al 
utilizar la int80, no necesitamos parametros extra, para estas funcionalidades extra.
\subsection{Reboot}
Luego de probar soluciones sucias, como hacer que el procesador triple-faultee y 
se reinicie la pc. Encontr�mos la solucion de enviar la se�al de reset desde el 
controlador de teclado en:
\begin{center}
http://wiki.osdev.org/Reboot
\end{center}
\subsection{Paginas web utilizadas}
En general leimos mucho de:
\begin{center}http://wiki.osdev.org/Main\_Page ( y sus foros )\\
http://www.jamesmolloy.co.uk/tutorial\_html/index.html\\
http://www.osdever.net/tutorials/view/brans-kernel-development-tutorial\\
\end{center}
\end{document}
